\section{Списочная теорема Брукса.}
\begin{theorem}[Визинг, 1976]
    Пусть $d \ge 3$, $G$ --- связный граф, отличный от $K_{d+1}$, $\Delta(G) \le d$. Тогда $\ch(G) \le d$.
\end{theorem}
\begin{proof}
    Пусть каждой вершине $v \in V(G)$ соответствует список $L(v)$, причем $l(v) \ge d$.

	\begin{itemize}
		\item Если $G$ --- не лес Галлаи, по теореме Бородина он раскрашивается.
		\item Пусть $G$ --- лес Галлаи. По условию $G$ не двусвязен, поэтоому его блоки точно отличны от $K_{d+1}$.

			Посмотрим на крайний блок $B$ и его вершину $b$, не являющуюся точкой сочленения. Так как этот блок не является полным подграфом:
			\[
			l(b) \ge d > d_{B}(b) = d_{G}(b)
			.\] 
			Значит, вершина $b$ избыточна, по лемме \ref{lm:coloring_5} существует искомая раскраска.
	\end{itemize}

\end{proof}
