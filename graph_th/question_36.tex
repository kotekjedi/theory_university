\section{Две леммы о d-раскрасках (о избыточной вершине и о удалении вершины с сохранением связности).}

\begin{definition}
    Граф $G$ называется \selectedFont{$d$-раскрашиваемым}, если для любого набора списков $L$, удовлетворяющего условию $l(v) \ge d_G(v)$ для каждой вершины $v \in V(G)$, существует правильная раскраска вершин в цвета из списков.

	Список цветов, удовлетворяющий указанному  условию, будем называть \selectedFont{$d$-списком}.
\end{definition}

\begin{definition}
    Назовем вершину $v \in V(G)$ \selectedFont{нормальной}, если $l(v) \ge d_G(v)$, \selectedFont{избыточной}, если $l(v) > d_{G}(v)$.
\end{definition}

\begin{lemma}\label{lm:coloring_5}
	Пусть $ G$ --- связный граф, $L$ --- $d$-список, в котором вершина $a$ избыточная. Тогда существует правильная раскраска вершин графа $G$ в соответствии со списком $L$. 
\end{lemma}
\begin{proof}
    Индукция по количеству вершин.
	\begin{description}
		\item [База:] граф с одной избыточной вершиной, очевидно.
		\item [Переход:]  пусть мы уже доказали утверждение для графов с меньшим числом вершин.

			Рассмотрим граф $G-a$. Пусть $G_1, \ldots , G_k$ --- все компоненты графа $G-a$. В каждом графе $G_i$ должна быть вершина $a_i$, смежная с $a$.

			Рассмотрим отдельно граф $G_i$ с исходными списками вершин. Тогда $a_i$ станет избыточной, так как $d_{G_i}(a) \le d_G(a_i) - 1 \le l(a_i) - 1$.

			По индукционному предположению вершины всех $G_i$ можно покрасить в соответствии со списками. Так как $a$ избыточная, мы можем раскрасить ее в какой-то цвет из списка $L(a)$, не нарушив правильности раскраски. 
	\end{description}
\end{proof}

\begin{lemma}\label{lm:coloring_6}
	Пусть $G$ --- связный граф, $L$ --- $d$-список. Предположим, что существуют две смежные вершины $a$ и $b$ такие, что граф $G-a$ связен и $L(A) \not\subset L(b)$. Тогда существует правильная раскраска вершин графа $G$ в соответствии со списком $L$.
\end{lemma}

\begin{proof}
    Пусть $1 \in L(A) \setminus L(B)$.

	В связном графе $G-a$ из всех списков вершин множества $N_{G}(A)$, содержащих цвет $1$, удалим этот цвет, остальные оставим без изменений. Получим новые списки $L'(v)$ графа $G-a$.

	Все вершины графа $G-a$ нормальны: вершины не из $N_G(a)$ не изменились, а для $v \in N_G(a)$ имеем
	\[
	l'(v) \ge l(v) - 1 \ge d_G(v) - 1 = d_{G-a}(v)
	.\] 

	Так как $1 \notin L(b)$, $l'(b) = l(b)$. Так как $d_{G-a)(b) = d_G(b)-1}$, вершина $b$ избыточная.

	По лемме \ref{lm:coloring_5} существует правильная раскраска вершин графа $G-a$ в цвета из $L'$. Далее докрашиваем $a$ в цвет $1$, получаем правильную раскраску вершин графа $G$ в цвета списка $L$.
\end{proof}

