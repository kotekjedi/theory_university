\section{Хроматический многочлен и блоки. Кратность корня 1 хроматического многочлена графа.}

\begin{lemma}\label{lm:coloring_14}
    Пусть $G$ --- связный граф с $n$ блоками $B_1, \ldots , B_n$. Тогда 
	\[
	\chi_G(k) = \left( \frac{1}{k} \right) ^{n-1} \cdot \prod_{i=1}^{n} \chi_{B_i} (k)
	.\] 
\end{lemma}
\begin{proof}
    Индукция по количеству блоков.
	\begin{description}
		\item[База:] для двусвязного графа, очевидно, это один блок.
		\item[Переход:] Пусть $n \ge 2$. НУО, $B_n$ --- крайний блок, содержащий ровно одну точку сочленения $a$.

			В графе $G' = G - \Int(B_n)$ ровно на один блок меньше, так как нет $B_n$. По индукционному продолжению для $G'$ :
			\[
			\chi_{G'}(k) = \left( \frac{1}{k} \right) ^{n-2} \cdot \prod_{i=1}^{n-1} \chi_{i}(k)
			.\] 
			Рассмотрим любую правильную раскраску $\rho$ графа $G'$ в $k$ цветов. Попробуем покрасить вершины $B_n$ c соблюдением правильности.

			Единственное ограничение --- цвет вершины $a$ уже зафиксирован, поэтому раскрасок в $k$ раз меньше.

			Следовательно, $\chi_G(k) = \frac{1}{k} \cdot  \chi_{G'}(k) \cdot \chi_{B_n}(k)$.
	\end{description}
\end{proof}
\begin{theorem}
    Пусть  $G$ --- связный граф с более чем одной вершиной. Тогда $1$ --- корень многочлена $\chi_G(k)$ кратности, равной количеству блоков графа $G$.
\end{theorem}
\begin{proof}
	Так как в каждом блоке хотя бы две вершины, достаточно доказать, что у хроматического многочлена графа без точек сочленения число $1$ является корнем кратности $1$, а далее применить лемму \ref{lm:coloring_14}. 

	Для $H \simeq K_2$ утверждение очевидно. Разберем второй вариант -- двусвязный граф. 

	$1$ точно корень,  так как раскрасить в один цвет двусвязный граф невозможно.

	Докажем, что $\chi'_{H}(1) \neq 0$, тогда мы покажем, что $1$ имеет кратность $1$.
	Для этого докажем, что для двусвязного графа $H$ на $m$ вершинах $\chi'_H(1) \neq 0$ и имеет такой же знак, как $(-1)^{m}$.

	\textbf{Индукция по $v(H)$.}
	
	\begin{description}
		\item[База:]
			Если $H$ --- полный граф на трех вершинах, то 
			\[
			\chi_{K_3}(k) = k(k-1)(k-2) \qquad \chi'_{K_3}(1) = 1(1-2) = -1
			.\] 
		\item[Переход:] Пусть $v(H)>3$. Тогда по теореме \ref{thm:connectivity_11} существует такое ребро $e \in E(H)$, что граф $H \cdot e$ двусвязен.

			По лемме \ref{lm:coloring_10} $\chi'_{H}(1) = \chi_{H-e}'(1) - \chi'_{H \cdot e} (1)$.

			Так как $v(H \cdot e) < e(H)$, если граф $H-e$ двусвязен, то уже доказано, что $\chi_{H-e}'(1)$ имеет тот же знак, что и $(-1)^{m}$.

			Если $H-e$ односвязен, то он имеет хотя бы два блока. Тогда для него верна лемма \ref{lm:coloring_12}.

			Так как хроматический многочлен каждого блока имеет корень $1$, причем для недвусвязного графа $H-e$ его хроматический многочлен имеет $1$ корнем кратности хотя бы $2$. И тогда $\chi_H'(1) = 0$, разность тоже не может быть равна нулю, а знак сохраняется из $H \cdot e$.
	\end{description}
\end{proof}
