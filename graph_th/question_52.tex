\section{Циклический обход границы.}
\begin{definition}
    Рассмотрим любую вершину $a$ плоского графа $G$ и упорядочим выходы ребер из $a$ по часовой стрелке. Два ребра, выходы которых соседние в 
	этом порядке, будем называть \selectedFont{соседними в вершине $a$ }. 
\end{definition}
\begin{lemma}\label{lm:planar_2}
    Пусть $ab_1$ и $ab_2$ --- два соседних ребра в вершине $a$. Тогда $ab_1$ и $ab_2$ лежат в границе некоторой грани.
\end{lemma}
\begin{proof}
    Вершины $b_1$ и $b_2$ можно соединить ломаной вдоль $b_1ab_2$, не пересекающей изображения $G$. Поэтому, ребра $ab_1$ и $ab_2$ лежат на границе некоторой грани.
\end{proof}
\subsection{Циклический обход границы}
Пусть $G$ --- плоский граф, $d \in F(G)$, $x_1x_2 \in E_d$.

Пройдем по ребру $x_1x_2$ от $x_1$ до $x_2$. НУО справа по ходу движения расположена грань $d$.

Повернем в вершине  $x_2$ направо до выхода соседнего ребра $x_2x_3$. Если $d_G(x_2) = 1$, то $x_1 = x_3$, это не проблема. Также $x_2x_3 \in E_d$.

Пройдем по этому ребру от $x_2$ к $x_3$, справа опять будет расположена грань $d$. И так далее. В итоге мы вернемся на ребро $x_1x_2$, при этом в вершину $x_1$ мы могли приходить и по другому ребру.

Мы получили замкнутый циклический путь, см. рис. \ref{fig:cycle}.
\begin{figure}[ht]
    \centering
	\begin{subfigure}{0.48\textwidth}
		\centering
		\incfig{cycle}
		\caption{}
		\label{fig:cycle}
	\end{subfigure}
	\hfill
	\begin{subfigure}{0.48\textwidth}
		\centering
		\incfig{cycle-2}
		\caption{cycle-2}
		\label{fig:cycle-2}
	\end{subfigure}
\end{figure}

Пусть получился циклический маршрут $Z = x_1 x_2 \ldots x_k$. Рассмотрим вершину $x_i$. По построению $Z$ обходит вокруг $x_i$ --- пусть против часовой стрелки. 

Пусть мы вышли из $x_i$ по ребру $x_ix_{i+1}$, вернулись по ребру $x_{j-1} x_j  = x_{j-1}x_i$, см. рис. \ref{fig:cycle-2}.

Тогда сектор между выходами ребер $x_i x_{i+1}$ b $x_i x_{j-1}$ не принадлежит грани $d$.

Следовательно, $Z$ проходит все ребра из $E_d$, инцидентные вершине $x_i$. Поскольку это верно для любой входящей в $Z$ вершины, этот маршрут обходит в точности все ребра одной из компонент графа $B(d)$.

Обозначим за $Z(U)$ такой маршрут для компоненты $U$, а через $Z(d)$ --- объединение построенных маршрутов для всех компонент $B(d)$.

Если маршрут $Z(d)$ проходит ребро $e$ дважды, то в разных направлениях. Значит, по обе стороны от $e$ расположена грань $d$, то есть $e$ --- внутренне ребро $d$.

Пусть $e$ --- внутреннее ребро грани $d$. Тогда при проходе по $e$ в любом из направлений справа будет расположена грань $d$. Поэтому, маршрут $Z)d)$ дважды пройдет $e$ в обоих направлениях.
