\section{Независимые множества, паросочетания и покрытия в графе. Теорема Галлаи.}
\begin{definition}
	Множество вершин $U \subset V(G)$ называется \selectedFont{независимым}, если никакие две его вершины не смежны. Обозначим через $\alpha(G)$ количество вершин в максимальном независимом множестве графа $G$.
\end{definition}

\begin{definition}
    Множество ребер $M \subset E(G)$ называется \selectedFont{паросочетанием}, если никакие его два ребра не имеют общей вершины. Обозначим через $\alpha'(G)$ количество ребер в максимальном паросочетании графа $G$.
\end{definition}

\begin{definition}
    Паросочетание $M$ графа $G$ называется \selectedFont{совершенным}, если оно покрывает все вершины графа.
\end{definition}

\begin{definition}
    Будем говорить, что множество вершин $W \subset V(G)$ \selectedFont{покрывает} ребро $e \in E(G)$, если существует вершина $w \in W$, инцидентная $e$. Будем говорить, что множество ребер $F \subset E(G)$ \selectedFont{покрывает} вершину $v \in V(G)$, если существует ребро $f \in F$, инцидентное $v$.
\end{definition}

\begin{definition}
    Множество вершин $W \subset V(G)$ называется \selectedFont{вершинным покрытием}, если оно покрывает все ребра графа. Обозначим через $\beta(G)$ количество вершин в минимальном вершинном покрытии графа $G$.
\end{definition}

\begin{definition}
    Множество ребер $F \subset E(G)$ называется \selectedFont{реберным покрытием}, если оно покрывает все вершины графа. Обозначим через $\beta'(G)$ количество ребер в минимальном реберном покрытии графа $G$.
\end{definition}

\begin{lemma}\label{lm:matching_1}
    \begin{enumerate}
		\item $U \subset V(G)$ --- независимое множество, согда $V(G) \setminus U$ --- вершинное покрытие.
		\item $\alpha(G) + \beta(G) = v(G)$.
    \end{enumerate}
\end{lemma}
\begin{proof}
    \begin{enumerate}
		\item Если $U$ --- независимое множество, то все ребра из этих вершин выходят в $V(G) \setminus U$, значит все ребра покрываются $V(G) \setminus U$. Если $V(G) \setminus U$ --- вершинное покрытие, ребер внутри $U$ быть не может, следовательно $U$ --- независимое множество.
		\item Применяем первый пункт для максимального независимого множества и минимального вершинного покрытия.
    \end{enumerate}
\end{proof}

\begin{theorem}[Галлаи, 1959]
	Пусть $G$ --- граф с $\delta(G) > 0$. Тогда $$\alpha'(G) + \beta'(G) = v(G)$$.
\end{theorem}
\begin{proof}
    Докажем неравенство в обе стороны.
	\begin{itemize}
		\item Пусть $M$ --- максимальное паросочетание, $U$ --- множество не покрытых $M$ вершин графа. $\lvert U \rvert = v(G) - 2 \alpha'(G)$.

			Так как $\delta(G) > 0$, можно выбрать множество $F$ из $\lvert U \rvert $ ребер, покрывающее $U$. 

			Тогда $M \cup F$ --- покрытие,
			\[
			\beta'(G) \le \lvert M \cup F \rvert \alpha'(G) + v(G) - 2 \alpha'(G)
			.\] 
			Из этого получаем неравенство $\alpha'(G) + \beta'(G) \le v(G)$.
		\item Пусть $L$ --- минимальное реберное покрытие, $\lvert L \rvert = \beta'(G)$, Рассмотрим подграф $H = G(L)$, порожденный ребрами покрытия.

			Все компоненты связности в $H$ --- звезды, иначе $L$ не минимально. В каждой компоненте можем выбрать только одно ребро в паросочетание.

			Следовательно, $\alpha'(G) \ge c(H)$ и $\beta'(G) = \lvert L \rvert = e(H) \ge v(H) - c(H) = v(G) - c(H)$. Сложим два неравенства и получим $\alpha'(G) + \beta'(G) \ge v(G)$.
	\end{itemize}
\end{proof}
