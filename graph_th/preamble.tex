\usepackage [utf8] {inputenc}
\usepackage [T2A] {fontenc}
\usepackage[english, russian]{babel}

\usepackage{amssymb, amsthm, amsfonts}
\usepackage{amsmath}
\usepackage{mathtools}
\usepackage{needspace}
\usepackage{enumitem}

% разметка страницы и колонтитул
\usepackage{geometry}
 \geometry{
 a5paper,
%  total={148mm,210mm},
 left=10mm,
 top=15mm,
 right=10mm,
 bottom=10mm,
 }

\let\lectionnumber=1
\usepackage{fancybox,fancyhdr}
\fancyhf{}
% \fancyhead[R]{\thepage}
% \fancyhead[L]{\thesection}
\fancyhead[RO,LE]{\thepage}
\fancyhead[LO]{\bf\rightmark}
% \fancyhead[L]{Лекция \lectionnumber}
\fancyfoot{}
\fancyhfoffset{0pt}
\addtolength{\headheight}{13pt}
\pagestyle{fancy}


% Отступы
\setlength{\parindent}{0ex}
\setlength{\parskip}{3mm}

\usepackage{graphicx}
\usepackage{subcaption}
\usepackage{hyperref}

\usepackage{import}
\usepackage{xifthen}
\usepackage{pdfpages}

\newcommand{\incfig}[1]{%
    \def\svgwidth{\columnwidth}
    \import{./figures/}{#1.pdf_tex}
}
\pdfsuppresswarningpagegroup=1

\renewcommand{\proofname}{\underline{\it proof}}
\renewenvironment{proof}
{ \underline{\it proof.}  }
{ \hspace{\stretch{1}}  $\square$ }

\usepackage{xifthen}
\makeatother
\def\@lecture{}%
\newcommand{\lecture}[2]{
	\def\@lecture{Лекция #1: #2}%
    \subsection*{\@lecture}
	% \let\lectionnumber=#1
	% Лекция #1: \mbox{#2}\\
}
\makeatletter

\usepackage{xcolor}
\definecolor{Aquamarine}{cmyk}{50, 0, 17, 100}
\definecolor{ForestGreen}{cmyk}{76, 0, 76, 45}
\definecolor{CarnationPink}{cmyk}{0, 0.349, 0.2118, 0}
\definecolor{Pink}{cmyk}{0, 100, 0, 0}
\definecolor{Cyan}{cmyk}{56, 0, 0, 100}
\definecolor{Gray}{gray}{0.3}

 % Цвета для гиперссылок
\definecolor{linkcolor}{HTML}{3f888f} % цвет ссылок
\definecolor{urlcolor}{HTML}{af0000} % цвет гиперссылок
 
\hypersetup{pdfstartview=FitH,
	linkcolor=linkcolor,urlcolor=urlcolor, colorlinks=true}

\theoremstyle{definition}
\newtheorem*{definition}{\underline{{\bf def}}}
\newtheorem*{theorem}{\underline{{\bf thm}}}

\theoremstyle{plain}
\newtheorem{corollary}{{\bf cor}}
\newtheorem{lemma}{{\bf lm}}
\newtheorem*{statement}{{\bf st}}
\newtheorem*{property}{{\bf prop}}

\theoremstyle{remark}
\newtheorem*{name}{\underline{name}}
\newtheorem*{remark}{\underline{remark}}
\newtheorem*{comment}{\underline{comment}}
\newtheorem*{note}{\underline{note}}


\renewcommand{\o}{o}
\renewcommand{\O}{\mathcal{O}}

\renewcommand{\le}{\leqslant}
\renewcommand{\ge}{\geqslant}

\newcommand{\N}{\mathbb{N}}
\newcommand{\Z}{\mathbb{Z}}
\newcommand{\R}{\mathbb{R}}
\newcommand{\RR}{\mathfrak{R}}
\newcommand{\T}{\mathfrak{T}}
\renewcommand{\S}{\mathfrak{S}}
\newcommand{\G}{\mathcal{G}}

\newcommand{\dist}{\operatorname{dist}}
% \newcommand{\ch}{\operatorname{ch}}
\newcommand{\Int}{\operatorname{Int}}
\newcommand{\Part}{\operatorname{Part}}
\newcommand{\Bound}{\operatorname{Bound}}
\def\selectedFont#1{\textsf{#1}}
