\section{Теорема Визинга.}
\begin{definition}
    Через $\mu(G)$ обозначим \selectedFont{максимальную кратность} ребра графа $G$, то есть максимум $e_G(\{x\}, \{y\})$ для всех пар $x, y \in V(G)$.
\end{definition}
\begin{theorem}[Визинг, 1964]
    Для любого графа $G$ выполнено $\Delta(G) \le \chi'(G) \le \Delta(G) + \mu(G)$.
\end{theorem}
\begin{proof}
    Пусть $\mu = \mu(G), \Delta= \Delta(G)$. Достаточно доказать существование правильной раскраски ребер $G$ в  $\Delta + \mu$ цветов.

	Рассмотрим $(\Delta+\mu)$-оптимальную раскраску $\rho$ ребер $G$.
	Предположим, что эта раскраска неправильная.

	Тогда существует вершина $u$ и цвет $ i_1$, который дважды представлен в вершине $u$. Так как $d_{G}(u) < \Delta + \mu$, существует цвет $j$, не представленный в $u$.

	Пусть $uv_1 \in E(G)$ и $\rho(uv_1) = i_1$. Так как $d_G(v_1) < \Delta + \mu$, существует цвет $i_2$, не представленный в $v_1$.
	
	{\bf  Опишем один шаг построения. }

	Пусть различные цвета $i_1, \ldots , i_l $ и ребра $e_1, \ldots , e_l \in E(G)$ таковы, что $e_t = uv_t$, $\rho(e_t) = i_t$, цвет $i_{t+1}$ не представлен в вершине $v_t$.

	Будем говорить, что цвет $i_{t+1}$ \textit{выбран для вершины} $v_t$. Также вершины $v_1, \ldots , v_t $ не обязательно различны.

	Рассмотрим вершину $v = v_l$. Пусть в наборе $v_1, \ldots , v_l $ она встречается $m$ раз. Очевидно, $m \le \mu$.
	
	Тогда на предыдущих шагах мы рассматривали вершину $v$ и $m-1$ раз выбирали цвет, не представленный в этой вершине. Поскольку \[
	d_G(v_l) + m - 1 < \Delta + \mu
	,\] 
	существует цвет $i_{l+1}$, не представленный в вершине $v_l$ и не выбранный для нее на предыдущих шагах, его мы и выберем.

	Определим раскраску данного шага $\rho_l\colon \rho_l(e_s) = i_{s+1}$ при $s \in [1..l]$ и $\rho_l(e) = \rho(e)$ на остальных ребрах.
	\begin{statement}
	    Раскраска $\rho_l$ $k$-оптимальна. Цвет $i_{l+1}$ представлен в вершине $u$.
	\end{statement}
	\begin{proof}
	    Для вершин $x \notin \{u, v_1, \ldots , v_l\}$ цвета ребер не менялись, поэтому $\rho_l(x) = \rho(x)$.

		Рассмотрим вершину $w$, которая входит в  $\{v_1, \ldots , v_l\}$ ровно $n$ раз. Пусть $W = v_{s_1} = \ldots = v_{s_n}$.

		По построению все выбранные для вершины $w$ цвета $i_{s_1 + 1}, \ldots i_{s_n+1}$ различны, не представлены в вершине $w$ в раскраске $\rho$ и представлены в раскраске $\rho_l$.

		Цвета $i_{s_1}, \ldots , i_{s_n}$ представлены в вершине $w$ в раскраске $\rho$.

		Все отличные от $e_{s_1}, \ldots , e_{s_n}$ ребра, инцидентные $w$, не изменили свой цвет, поэтому остальные цвета одинаково представлены в $w$ в раскрасках $\rho$ и $\rho_l$, поэтому $\rho_l(w) \ge \rho(w)$.

		Рассмотрим вершину $u$. В результате перекрашивания инцидентных $u$ ребер $e_1, \ldots , e_l$ из их цветов исчез $i_1$, и появился $i_{l+1}$. Но так как цвет $i_1$ был представлен в $u$ в раскраске $\rho$ хотя бы дважды, он представлен и в раскраске $\rho$. 

		Тогда $\rho_l(u) \ge  \rho(u) $ и $\rho_l(G) \ge \rho(G)$, следовательно, раскраска $p_l$ оптимальна. Так как $\rho$ тоже  оптимальная, $\rho(G) = \rho_l(G)$, поэтому цвет $i_{l+1}$ был представлен в вершине $u$ и в раскраске $\rho$.
	\end{proof}
	Пусть $e_{l+1} = uv_{l+1}$ --- ребро цвета $\rho(e_{l+1}) = \rho(G)$. 

	{ \bf Так мы завершили еще один шаг. }

	Поскольку у $u$ конечное число соседей, на некотором шаге мы получим $i_{m+1} = i_k$. То есть $v_m$ не совпадает с $v_{k-1}$ (иначе мы выбрали бы $i_{m+1} \neq i_k$ ). Так как в вершинах $v_{k-1}$ и $v_m$ в раскраске $\rho$ не представлен цвет $i_{k}$, а в $v_k$ представлен, все три вершины $v_{k-1}, v_k, v_m$ различны.

	Рассмотрим $(\Delta+\mu)$-оптимальные раскраски $\rho_{k-1}$ и $\rho_m$ (считаем, что $\rho_0 = \rho$):
\begin{itemize}
	\item В обеих раскрасках в вершине $u$ дважды представлен цвет  $i_k$.
	\item Цвет $j$ не представлен в вершине $u$ ни в одной из раскрасок.
\end{itemize}
\begin{figure}[ht]
    \centering
	\begin{subfigure}{0.3\textwidth}
		\centering
		\incfig{vising-theorem}
		\caption{Раскраска $\rho$}
		\label{fig:vising-theorem}
	\end{subfigure}
	\hfill
	\begin{subfigure}{0.34\textwidth}
		\centering
		\incfig{vising-theorem-2}
		\caption{Раскраска $\rho_k$}
		\label{fig:vising-theorem-2}
	\end{subfigure}
	\hfill
	\begin{subfigure}{0.34\textwidth}
		\centering
		\incfig{vising-theorem-3}
		\caption{Раскраска $\rho_m$}
		\label{fig:vising-theorem-3}
	\end{subfigure}
\end{figure}

Пусть $E_s$ --- множество всех ребер цвета $s$ в раскраске $\rho_{k-1}$, $E_s' $ --- множество всех ребер цвета $s$ в раскраске $\rho_m$. $H = G(E_{i_k} \cup E_j')$ и $H' = G(E_{i_k}' \cup E_j')$.

По лемме \ref{lm:coloring_11} из оптимальности раскрасок следует, что содержащие вершину $u$ компоненты связности графов $H$ и $H'$ --- простые циклы нечетной длины.

Тогда $d_{H}(v_k) = 2$ : из $v_k$ выходит ребро $uv_k$ цвета $\rho_{k-1}(uv_k) = i_k$ и ребро цвета $j$. Для всех ребер $e$ цикла $H$, кроме $uv_k$ цвет $\rho_{k-1}(e) = \rho_m(e)$, поэтому $d_{H'}(v_k) = d_{H}(v_k) - 1 = 1$. 

Вершины $v_k$ и $u$ лежать в одной компоненте связности $H'$, которая должна быть нечетным циклом. Противоречие. 

Следовательно, $\rho$ --- искомая правильная раскраска в $\Delta + \mu$ цветов.
\end{proof}
