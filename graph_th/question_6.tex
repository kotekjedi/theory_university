\section{Теорема Татта о существовании регулярного графа степени k с обхватом g.}
\begin{definition}
	\selectedFont{Обхват графа} $G$ (обозначение  $g(G)$ ) --- длина наименьшего цикла в графе $G$.
\end{definition}
\begin{theorem}[Тaтт]
	Пусть $k, g, n \in \N$, причем $k, g \ge 3$, $kn$ четно и 
	 \[
	n > \frac{k(k-1)^{g-1} -2}{k-2}
	.\] 
	Тогда существует регулярный граф $G$ степени $k$ с $g(G) = g$ и $v(G) = n$.
\end{theorem}
\begin{proof}
		Пусть $\G(n, g, k)$ --- множество всех графов на $n$ вершинах с обхватом $g$ и максимальной степенью вершин не более $k$.

		Пусть $v_{<k}(G)$ --- количество вершин степени менее $k$ в графе $G$, $\dist_{<k}(G)$ --- максимальное из расстояний между парами вершин степени менее $k$ в графе $g$, при $v_{<k}(G) < 2$ положим $\dist_{<k}(G) = 0$.

		Если $n > g$, $\G(n, g, k) \neq \varnothing$, например, есть граф из цикла на $g$ вершинах и нескольких изолированных вершинах.

		Будем выбирать в $\G(n, g, k)$ граф следующим образом:
			\begin{enumerate}
				\item сначала возьмем все графы с максимальным количеством ребер,
				\item затем из них выберем графы с максимальным $v_{<k}$,
				\item из оставшихся выберем граф $G$ с максимальным $\dist_{<k}(G)$.
			\end{enumerate}

		{\bf Докажем, что $G$ --- регулярный граф степени $k$.}
		\begin{itemize}
			\item Пусть не так. Рассмотрим пару его максимально удаленных вершин степени менее $k$. Пусть это $x$ и  $y$ (возможно $x = y$).
			\item Если $\dist_G(x, y) \ge g-1$, то соединим $x$ и $y$ и получим граф $G' \in \G(n, g, k)$  с $e(G') > e(G)$, а такого не должно быть.
				Следовательно, $\dist_{G}(x, y) \le g-2$.
			\item Так как степени $x$ и $y$ меньше $k$, а степени всех остальных не больше $k$, то на расстоянии не более $g-1$ от $y$ находится не более чем $\frac{(k-1)^{g}-1}{k-2}$ вершин, а на расстоянии не более $g-2$ от $x$ не более $\frac{(k-1)^{g-1}-1}{k-2}$ вершин.
			\item По условию теоремы существует такая вершина $z$, что $\dist(x, z) \ge g-1$ и $\dist(y,z) \ge g$. % TODO: почему?
			\item Так как $\dist_{G}(x, y) \le g-2$, степень $d_G(z) = k \ge 3$. Следовательно, есть ребро $zu \in  E(G)$, через которое проходят не все простые циклы длины $g$ графа $G$. Тогда $g(G-zu) = g(G) = g$. % TODO: почему?
			\item $d_G(u) = k$, так как:
				 \[
				\dist_G(y, u) \ge \dist_G(y, z) -1 \ge g-1 > \dist(x,y) = dist_{<k}(G)
				.\] 
			\item Пусть $G' = G - zu + zx$.  $g(G') = g$, $e(G') = e(G)$, $d_{G'}(x) = d_{G}(x) + 1$, $d_{G'}(u) = d_{G}(u) -1 = k-1$, степени остальных вершин совпадают. Итого, $G' \in \G(n, g, k)$.
			\item Заметим, что $v_{<k}(G') \ge v_{<k}(G)$. По алгоритму выбора графа $G$ должно быть равенство, поэтому $d_{G'}(x) = k$ и $d_G(x) = k-1$.
			\item Так как $kn$ четно, вершина $x$ не может быть единственной вершиной степени меньше $k$ в графе $G$, следовательно, $x \neq  y$.
		\end{itemize}
		{\bf Докажем, что $\dist_{G'}(y,u) > \dist_{G}(y,x)$.}
		\begin{itemize}
			\item Найдем $yu$-путь  $P$, который реализует расстояние между $y$ и $u$ в $G'$.
			\item Если $P$ проходит  только про ребрам $G$, то
				\[
				\dist_{G'}(y,u) = \dist_G(y,u) \ge g-1 > \dist_G(y,x)
				.\] 
			\item Следовательно, $P$ проходит по новому ребру $zx$. Тогда $P$ содержит путь по ребрам графа $G$ от $y$ до  $x$ или $z$ и само ребро $zx$. 

				И, так как $\dist_G(y,z) \ge g > \dist_G(y,z)$:
				\[
				\dist_{G'}(y,u) \ge \min\bigl( \dist_G(y, x) + 1, \dist_G(y,z) +1) > \dist_G(y,x)
				.\] 
			\item Таким образом,
				\[
				\dist_{<k}(G') \ge \dist_{G'}(y,u) > \dist_G(y,x_ = \dist_{<k}(G)
				.\] 
				Противоречие. Значит, $G$ --- $k$-регулярный граф.
				% TODO: было бы круто добавить картинку
		\end{itemize}
\end{proof}
