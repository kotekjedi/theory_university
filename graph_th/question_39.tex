\section{k-критические графы. Простейшие свойства.}
\begin{definition}
    Назовем граф \selectedFont{$k$-критическим}, если $\chi(G) = k$, но $\chi(H) < k$ для любого собственного подграфа $H$ графа $G$.
\end{definition}
\begin{lemma}\label{lm:coloring_7}
    Если $G$ ---  $k$-критический граф, то $\delta(G) \ge k-1$.
\end{lemma}
\begin{proof}
    Пусть $a \in V(G)$, $d_G(a) \le k-2$. По определению $\chi(G-a) \le k-1$.

	Покрасим граф $G-a$ в $k-1$ цвет, так как степень $a$ в исходном графе меньше $k-1$, мы сможем докрасить ее и получить раскраску в $k-1$ цвет. Противоречие. 
\end{proof}

\begin{lemma}\label{lm:coloring_8}
    Пусть $G$ --- $k$-критический граф, $S \subset V(G)$ --- разделяющее множество $\lvert S \rvert < k$. Тогда $G(S)$ --- не полный.
\end{lemma}
\begin{proof}
    Пусть $G(S)$ --- полный, $S = \{a_1, \ldots , a_m\}$, $\Part(S) = \{F_1, \ldots , F_n\}$, $G_i = G(F_i)$.

	Так как $G_i$ --- собственный подграф $G$, то $\chi(G_i) \le k-1$, пусть $\rho_i$ --- правильная раскраска $G_i$ в $k-1$ цвет.

	Так как вершины $S$ попарно смежны в $G_i$, то все цвета  $\rho(a_1), \rho(a_2), \ldots , \rho(a_m)$ различны. Теперь согласуем раскраски в $G_1, \ldots , G_n$ и получим общую раскраску в $k-1$ цвет для вершин графа $G$. Противоречие. 
\end{proof}
