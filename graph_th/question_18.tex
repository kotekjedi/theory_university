\section{Алгоритм разбиения графа на блоки.}
Пусть $U_1, \ldots , U_k$ --- все  компоненты связности графа $G-a$, $G_i = G(U_i \cup \{a\})$. Разрежем граф  $G$ на графы $G_1, \ldots G_k$.

\begin{lemma}\label{lm:connectivity_4}
	\begin{enumerate}
		\item Пусть $b \in U_i$. Тогда $b$ разделяет вершины $x, y \in  U_i$ d $G_i$, согда $b$ разделяет их в $G$.
		\item Все точки сочленения графов $G_1, \ldots , G_k$ --- в точности все точки сочленения графа $G$, кроме $a$.
	\end{enumerate}
\end{lemma}
\begin{proof}
    \begin{enumerate}
		\item Если $G-b$ не содержит $xy$-путь, то его нет и в $G_i - b$.
		
			Наоборот, пусть $x$ и $y$ лежат в разных компонентах связности графа $G_i -b$. НУО можно считать, что компонента связности $W \ni x$ не содержит $a$. Тогда $W$ --- компонента связности графа $G-b$, следовательно, в $G-b$ тоже не было $xy$-пути.
		\item Так как $G_i - a$ --- компонента графа $G-a$, вершина $a$ не является точкой сочленения ни в одном  из графов $G_1, \ldots , G_k$.

			Любая другая точка сочленения графа $G$ лежит ровно в одном из графов $G_1, \ldots , G_k $ и является в нем точкой сочленения по прошлому пункту.

			Так же из прошлого пункта следует, что других точек сочленения в графах $G_1, \ldots G_k $ нет.
    \end{enumerate}
\end{proof}
\subsection{Алгоритм разбиения связного графа на блоки}
\begin{itemize}
	\item Выберем точку сочленения $a$ и разрежем по ней $G$: заменим граф $G$ на полученные $G_1, \ldots , G_k$.
	\item Каждый следующий шаг берем один из имеющихся графов, выбирает точку сочленения и разрезаем по ней.
	\item И так далее, пока хотя бы один из полученных графов имеет точку сочленения.
\end{itemize}

\begin{theorem}
    В результате описанного алгоритма вне зависимости от порядка действий получатся блоки графа $G$.
\end{theorem}
\begin{proof}
	По лемме \ref{lm:connectivity_4} мы вне зависимости от порядка проведем разрезы только по всем точками сочленения и только по ним.

	Пусть $B$ --- блок графа $G$. Тогда в графе $G$ множество $V(B) $ не было разделено ни одной из точек сочленения. Тогда по первому пункту леммы \ref{lm:connectivity_4} множество $V(B)$ не было разрезано алгоритмом.

	Так как в результате алгоритма получились индуцированные подграфы графа $G$, один из них обозначим за $H$ --- надграф  $B$.

	Если $H \neq B$, то рассмотрим вершину $c \in V(H) \setminus V(B)$. 
	В графе $G$ существует точка сочленения $a$, отделяющая $c$ от $V(B)$. Тогда по лемме  \ref{lm:connectivity_4} при разрезе по $a$ вершина $c$ была отделена от блока $B$. Противоречие. 
    
\end{proof}
