\section{Конструкция графа с произвольным хроматическим числом без треугольников.}
\begin{definition}
    \selectedFont{Кликовое число} графа $G$ (обозначение $\omega(G)$ ) --- это количество вершин в максимальной клике.
\end{definition}

\begin{theorem}[Мычельский, 1955]
	Для любого $ k \in \N$ существует граф $G$, удовлетворяющий условиям $\chi(G) = k$, $g(G) \ge 4$.
\end{theorem}
\begin{proof}
	Для $k=1$ и $k=2$  подойдут полные графы $K_1$ и $K_2$.

	Построим следующие графы $G_3, G_4 , \ldots $ без треугольников с $\lambda(G_k) = k$.

	Пусть построен граф $G_k$, причем $V(G_k) = \{u_1, \ldots  ,u_n \}$.
	Этот граф будет частью графа $G_{k+1}$, в котором будут добавлены вершины $v_1 , \ldots  , v_n , w$. 

	Ребра между новыми вершинами проведем так: $v_i$ будет смежна со всеми вершинами из $N_{G_k}(ui)$ и только с ними, а $w$ --- со всеми вершинами $v_1 , \ldots  , v_n$ и только с ними (см. рис. \ref{}). 
	\begin{figure}[ht]
		\centering
		\incfig{mycielski-theorem}
		\caption{}
		\label{fig:mycielski-theorem}
	\end{figure}

	Понятно, что треугольников в графе $G_{k+1}$ нет. 

	Далее заметим, что $\chi(G_{k+1}) \le  k + 1$: если $\rho$ -- правильная раскраска вершин $G_{k+1}$ в $k$ цветов, то можно продолжить ее на $G_{k+1}$, использовав только один дополнительный цвет, для этого положим $\rho(v_i) = \rho(u_i)$ и $\rho(w)  = k+1$.

	Предположим, что $\chi(G_{k+1}) \le k_i$, и рассмотрим правильную раскраску $\rho$ вершин графа $G_{k+1}$ в $k$ цветов.

	НУО $\rho(w) = k$. Построим правильную раскраску $\rho'$ вершин $G_k$ в $k-1$ цвет, получим противоречие.

	 Для каждой вершины положим, $\rho' (u_i) = \rho(u_i)$, если $\rho(u_i) \neq k$, и $\rho'(u_i) = \rho(v_i)$, если $\rho(u_i) = k$. 

	 Так как вершины $v_1, \ldots , v_n$ смежны с вершиной $w$ цвета $k$, то их цвета отличны от $k$, следовательно, $\rho'\colon V(G_k) \to [1..k-1]$. 

	 Докажем правильность раскраски $\rho:$. Предположим противное, пусть $\rho' (u_i ) = \rho' (u_j )$, вершины $u_i$ и $u_j$ смежны.

	 Очевидно, хотя бы одна из них перекрашена, пусть это $u_i$, тогда $\rho'(u_i) = \rho(v_i)$. 

	 Мы перекрашивали только вершины, имеющие цвет $k$ в раскраске $\rho$, среди них не было смежных, следовательно, $\rho' (u_j ) = \rho(u_j )$.

	 По построению, из $u_j \in N_{G_k} (u_i )$ следует $u_j \in  N_{G_k} (v_i )$ и мы можем сделать вывод, что \[
	 \rho'(u_i) = \rho(v_i) \neq \rho(u_j) = \rho'(u_j)
	 .\] 
	 Противоречие. 

	 Таким образом, $\rho'$ --- правильная раскраска вершин графа $G_k$, противоречие. Следовательно, $\chi(G_{k+1}) = k + 1$.

\end{proof}

