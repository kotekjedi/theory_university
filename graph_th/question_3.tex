\section{Критерий существования гамильтонова цикла через связность.}
\begin{lemma}\label{lm:circle_3}
    Пусть граф $G$ гамильтонов. Тогда для любого множества $S \subset V(G)$
	выполняется неравенство $c(G - S) \le \lvert S \rvert$\footnote{$c(G)$ --- число компонент связности в графе $G$}.
\end{lemma}
\begin{proof}
	Пусть $c(G-S) = c$ и $U_1, \ldots , U_c$ --- компоненты связности графа $G - S$, $Z$ --- гамильтонов цикл графа $G$.

	Начнем обходить цикл $Z$, начиная с вершины из множества $S$. Пусть $s_i$ --- вершина, которая предшествует первому входу цикла в компоненту $U_i$.

	Все $s_i$ различны, причем принадлежат $S$, так как не могут входить ни в одну из компонент (иначе это одна компонента, а тогда вершина неправильная). Отсюда следует требуемое неравенство.
\end{proof}

\begin{theorem}[Хватал, Эрдёшь, 1972]
	Пусть $v(G) \ge 3$ и $\kappa(G) \ge \alpha(G)$\footnote{$\kappa(G)$ --- вершинная связность, $\alpha(G)$ --- размер максимального независимого множества}, тогда $G$ гамильтонов.
\end{theorem}
\begin{proof}
    \begin{itemize}
		\item Если в графе нет циклов, то есть $\kappa(G) == 1$. Тогда точно $\alpha(G) \ge 2$, так как вершин не меньше трех. По условию такой случай невозможен.
		\item Пусть $\kappa(G) = k$. Выберем цикл $C$ максимальной длины в графе $G$. Пусть  $C = v_1v_2\ldots v_n$\footnote{Считаем, что нумерация циклическая}.
		\item Пусть $C$ не гамильтонов. Рассмотрим компоненту связности $W$ графа $G - V(C)$. Заметим, что $N_G(W) \subset V(C)$.
				\begin{figure}[ht]
					\centering
					\begin{subfigure}{0.48\linewidth}
						\centering
						\incfig{thm-khvatal-ehrdesh}
						\caption{}
						\label{fig:thm-khvatal-ehrdesh}
					\end{subfigure}
					\begin{subfigure}{0.48\linewidth}
						\centering
						\incfig{thm-khvatal-ehrdesh-2}
						\caption{}
						\label{fig:thm-khvatal-ehrdesh-2}
					\end{subfigure}
				\end{figure}
		\item (a) Обозначим за $M = \{v_{i+1}\colon v_i \in N_{G}(W)\}$. Докажем, что $M \cap N_G(W) = \varnothing$.
			\begin{itemize}
				\item Пусть $v_i, v_{i+1} \in  N_G(W)$ и $w, w' \in  W$, $v_iw, v_{i+1}w' \in  E(G)$, $P$ --- $ww'$-путь по вершинам из $W$.
				\item Тогда можно удлинить цикл $C$ хотя бы на одно ребро, заменив ребро $v_iv_{i+1}$ на ребро $v_iw$, далее путь $P$, потом ребро $w'v_{i+1}$. Но цикл $C$ должен был быть максимальным. Противоречие. 
			\end{itemize}
			Это означает, что $N_G(W)$ отделяет непустое множество $M$ от $W$, следовательно $\lvert M \rvert = \lvert N_G(W) \rvert \ge k$
		\item (b)
			\begin{itemize}
				\item Теперь предположим, что вершины $v_{i+1}, v_{j+1} \in  M$ смежны. Пусть $w, w' \in W$ и $v_iw, v_jw' \in E(G)$ и $P$ --- $ww'$-путь по вершинам компоненты $W$.
				\item Рассмотрим цикл $Z$, проходящий сначала участок $v_{j+1}v_{j+2}\ldots v_i$ цикла $C$, затем ребро  $v_iw$, далее путь $P$ и ребро $w'v_j$, потом участок $v_jv_{j-1}\ldots v_{i+1}$ по циклу $C$ и ребро $v_{i+1}v_{j+1}$. Построенный цикл $Z$ длиннее $C$, противоречие.
			\end{itemize}
			Из этого следует, что $M \cup \{w\}$ --- независимое множество с $\lvert M \rvert + 1 > k$ вершин. А это противоречит условию $\alpha(G) \le k$. 
		\item В итоге, цикл $C$ должен быть гамильтоновым.
    \end{itemize}
\end{proof}
