\section{Крайние блоки}

\begin{definition}
    Назовем блок $B$ \selectedFont{крайним}, если он соответствует листу дерева блоков и точек сочленения.
\end{definition}
\begin{definition}
    \selectedFont{Внутренность} $\Int(B)$ блока $B$ --- множество всех его вершин, не являющихся точками сочленения в графе.
\end{definition}
\begin{itemize}
	\item Блок недвусвязного графа крайний, согда он содержит ровно одну точку сочленения.
	\item Внутренность некрайнего блока может быть пустой, а крайнего всегда непуста.
	\item Если у связного графа есть точки сочленения, то он имеет хотя бы два крайних блока.
	\item Если $B$ ---  блок графа $G$, $x \in \Int(B)$, то граф $G-x$ связен.
\end{itemize}
\begin{lemma}\label{lm:connectivity_3}
    Пусть $B$ --- крайний блок связного графа $G$ c $v(G) \ge 2$, $G' = G - \Int(B)$. Тогда граф $G'$ связен, а блоки $G'$ --- все блоки $G$, кроме $B$.
\end{lemma}
\begin{proof}
    Пусть $a \in V(B)$ --- точка сочленения, отрезающая крайний блок $B$ от остального графа $G$. Тогда $\Int(B)$ --- это одна из компонент связности графа  $G-a$, следовательно, сам граф $G'$ будет связен.

	Отличные от $B$ блоки графа $G$ --- подграфы $G'$, не имеют точек сочленения и являются максимальными подграфами $G'$ с таким свойством, так как были максимальными в $G$. Следовательно, они все --- блоки $G'$.

	Пусть $B'$ --- блок $G'$. Очевидно, что $v(G') \ge 2$, поэтому $B'$ содержит хотя бы одно ребро $e$, которое в графе $G$ лежит в некотором блоке $B^* \neq B$, так как блок максимальный по включению, $B^* = B'$.
\end{proof}
