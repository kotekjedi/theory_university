\chapter{Пути и циклы} 
\section{Введение}

\lecture{1}{15 feb}{\dag}
Все материалы на сайте \url{https://logic.pdmi.ras.ru/~dvk/MKN/graph_th}.

\begin{theorem}
	Связный граф $G$ --- эйлеров, согда степени всех вершин $G$ четны.
\end{theorem}

\begin{definition}[]
	\selectedFont{Гамильтонов путь} 
	\selectedFont{Гамильтонов цикл} 
	\selectedFont{Гамильтонов граф} 
\end{definition}

\begin{st}
    Пусть $n > 2$, $a_1 \ldots a_n$ --- максимальный путь (по ребрам) в графе $G$, причем $d_{G}(a_1) + d_{G}(a_{n}) \ge n$. Тогда в графе есть цикл длины $n$.
\end{st}
\begin{proof}
    Если $a_1$ и $a_{n}$ смежны, то $a_1a_2\ldots a_{n}$ --- искомый цикл.

	Иначе $N_{G}(a_1), N_{G}(a_{n}) \subset \{a_2, \ldots a_{n-1}\}$.
\end{proof}

\begin{theorem}[Критерий Оре]
	\begin{enumerate}
		\item Если для любых двух несмежных вершин $u, v \in V(G)$ выполняется 
			\[
			d_G(u) + d_{G}(v) \ge v(G) - 1
			,\] 
			то в графе $G$ есть гамильтонов путь.
		\item Если  $v(G) > 2$ и для любых двух несмежных вершин  $u, v \in V(G)$ выполняется
			\[
			d_{G}(u) + d_{G}(v) \ge v(G)
			,\] 
			то в графе $G$ есть гамильтонов цикл.
	\end{enumerate}
\end{theorem}
