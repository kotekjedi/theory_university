\section{Дерево блоков и точек сочленения и его свойства.}
\begin{definition}
    Построим двудольный граф $B(G)$, вершины которого --- точки сочленения $a_1, \ldots , a_n$ графа $G$, а вершины другой доли --- его блоки $B_1, \ldots, B_m$. Вершины $a_i$ и $B_j$ будут смежны, если $a_i \in V(B_j)$.

	Такой граф называется \selectedFont{деревом блоков и точек сочленения}.
\end{definition}

\begin{lemma}\label{lm:connectivity_2}
	Пусть $B_1$ и $B_2$ --- два разных блока графа $G$, а $P$ --- путь между ними в графе $B(G)$. Тогда точки сочленения графа $G$, отделяющие $B_1$ от $B_2$ --- это в точности те точки сочленения, что лежат на пути $P$. Остальные не разделяют даже объединение блоков пути $P$. 
\end{lemma}
\begin{itemize}
	\item
    Пусть $x$ ---  точка сочленения графа $G$, не лежащая на пути $P$, $H$ --- объединение всех блоков на пути $P$.

	Для любого блока $B$ на пути $P$ граф $B-x$ связен. Если $B$ --- не $B_1$ и не $B_2$, то в нем можно пройти между двумя точками сочленения, входящими в $P$, так как $x$ не входит в $P$. Поэтому $H-x$ --- связный граф.
\item

	Пусть $a$ ---  точка сочленения, лежащая на  $P$, и она входит в блоки $B_1'$ и $B_2'$ на пути $P$.

	Обозначим через $H_1$ объединение всех блоков на пути $P$ до $a$, через $H_2$ --- объединение всех блоков после $a$.

\begin{figure}[ht]
    \centering
    \incfig{block-tree}
    \caption{}
    \label{fig:block-tree}
\end{figure}

	Применим рассуждения первого пункта отдельно к $H_1$ и к $H_2$. Получаем, что $a$ не разделяет ни одного из них.

	С другой стороны, по лемме \ref{lm:connectivity_1} точка сочленения $a$ отделяет блок $B_1'$ от $B_2'$, а значит, $a$ отделяет $H_1$ от $H_2$, следовательно и $B_1$ от $B_2$.
\end{itemize}

\begin{theorem}
    \begin{enumerate}
		\item  Дерево блоков и точек сочленения --- это дерево, все листья которого соответствуют блокам.
		\item Точка сочленения $a$ разделяет два блока $B_1$ и $B_2$, согда $a$ разделяет $B_1$ и $B_2$ в $B(G)$.
    \end{enumerate}
\end{theorem}
\begin{proof}
    \begin{enumerate}
		\item Докажем первый пункт.
			\begin{description}
				\item[Связность.] Для любых двух вершин $B(G)$ рассмотрим путь $Q$ в $G$ между ними.

					Перестроим его в путь в $B(G)$ : участок пути $Q$, проходящий по одному блоку графа $G$, заменим на соответствующую блоку вершину в $B(G)$, переход $Q$ между различными блоками по лемме \ref{lm:connectivity_1} осуществляется через их общую точку сочленения --- вершину $B(G)$.
				\item[Дерево.] Пусть в $B(G)$ есть простой цикл $Z$. Рассмотрим подграф $H$ --- объединение всех блоков этого цикла. 

					Между любыми двумя входящими в  $ Z$ блоками есть два независимых пути в $B(G)$.

					По лемме \ref{lm:connectivity_2} граф $H$ не имеет точек сочленения, иначе они должны лежать на одновременно на двух путях по циклу.

					Следовательно, существует блок $B$, содержащий $H$, блоки цикла $Z$ --- собственные подграфы $B$, что невозможно.
				\item [Листья.] Если лист соответствует точкe сочленения $a$, то по лемме \ref{lm:connectivity_2} граф $G-a$ связен. Противоречие. 
			\end{description}
		\item Докажем второй пункт: в дереве $B(G)$ есть единственный путь между блоками $B_1$ и $B_2$, по лемме \ref{lm:connectivity_2} в точности точки сочленения с этого пути отделяют $B_1$ от $B_2$ в исходном графе $G$.
    \end{enumerate}
\end{proof}
