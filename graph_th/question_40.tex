\section{Теорема Галлаи о k-критических графах.}
\begin{theorem}[Галлаи, 1963]
    Пусть $k \ge 3$, $G$ --- $k$-критический граф. Пусть $V_{k-1}$ --- множество всех вершин графа $G$, имеющих степень $k-1$, а $G_{k-1} = G(V_{k-1})$. Тогда $G_{k-1}$ --- лес Галлаи.
\end{theorem}
\begin{proof}
	По лемме \ref{lm:coloring_7} $\delta(G) \ge k-1$. Будем считать, что $G_{k-1} \neq \varnothing$, иначе доказывать нечего.

	Предположим, что $G_{k-1}$ не лес Галлаи. Тогда этот граф имеет компоненту $G'$, у которой есть блок, отличный от полного графа и нечетного цикла. Пусть $V(G') = V'$.

	Для собственного подграфа $H = G - V'$ графа $G$ мы имеем $\chi(H) \le k-1$, так как $G$ --- $k$-критический.

	Пусть $\rho$ --- раскраска графа $H$ в  $k-1$ цвет. Рассмотрим любую вершину $x \in  V'$, пусть она имеет $n_x$ соседей  в $V(H)$. Поместим в список $L(x)$ в точности те цвета из $[1..k-1]$, что не встречаются среди $n_x$. Тогда длина списка $l(x) \ge k-1-n_x = d_{G'}(x)$.

	По теореме Бородина граф $G'$ является $d$-раскрашиваемым, следовательно, существует правильная раскраска $\rho^*$  графа $G'$ в цвета из построенных списков.

	Вместе $\rho$ и $\rho^*$ дают правильную раскраску вершин $G$ в $k-1$ цвет. Противоречие. Значит $G_{k-1}$ --- лес Галлаи.
\end{proof}
