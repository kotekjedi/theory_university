\section{Лемма о несвязной границе грани несвязного графа.}
\begin{lemma}\label{lm:planar_3}
	Для плоского графа $G$ выполнены следующие утверждения:
    \begin{enumerate}
		\item Если  $d \in F(G) $ и $B(d)$ несвязна, то разные компоненты связности графа $B(d) $ лежат в разных компонентах связности графа $G$.
		\item Граф $G$ несвязен, согда он имеет грань с несвязной границей.
    \end{enumerate}
\end{lemma}
\begin{proof}
    
\end{proof}
