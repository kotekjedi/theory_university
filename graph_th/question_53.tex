\section{Лемма о несвязной границе грани несвязного графа.}
\begin{lemma}\label{lm:planar_3}
	Для плоского графа $G$ выполнены следующие утверждения:
    \begin{enumerate}
		\item Если  $d \in F(G) $ и $B(d)$ несвязна, то разные компоненты связности графа $B(d) $ лежат в разных компонентах связности графа $G$.
		\item Граф $G$ несвязен, согда он имеет грань с несвязной границей.
    \end{enumerate}
\end{lemma}
\begin{proof}
	\begin{enumerate}
		\item Пусть $B_1$ и $B_2$ --- две компоненты $B(d)$. 

			Изображение $B_1$ ограничено и не пересекает других компонент $B(d)$. Следовательно, изображение $B_1$ можно отделить от изображения $B_2$ замкнутой ломаной в грани $d$, не пересекающей ребер $G$ (как в доказательстве теоремы Жордана). Значит, между $B_1$ и $B_2$ нет путь в графе $G$.
		\item Пусть граф несвязен, но все грани имеют связные границы. Тогда можно обойти все грани графа $G$, каждый раз переходя в грань, имеющую с предыдущей общую вершину или ребро. Но тогда $G$ связен. Противоречие. 

			По первому пункту следует обратное утверждение.
	\end{enumerate}    
\end{proof}
