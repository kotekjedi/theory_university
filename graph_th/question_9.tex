\section{Теорема Татта о совершенном паросочетании.}
\begin{definition}
    Для произвольного графа $G$ через $o(G)$ обозначим количество нечетных компонент связности графа $G$.
\end{definition}

\begin{theorem}[Татт, 1947]
	В графе $G$ существует совершенное паросочетание, согда для любого $S \subset V(G)$ выполняется условие $o(G-S) \le \lvert S \rvert$.
\end{theorem}
\begin{proof}
    \begin{description}
		\item[$ \implies$ ] Пусть $S \subset V(G)$, $M$ --- совершенное паросочетание. Тогда одна из вершин каждой нечетной компоненты связности графа $G-S$ должна быть соединена с вершиной из $S$ ребром паросочетания $M$, при этом все такие вершины различны, так как входят в паросочетание только один раз.
		\item[$ \impliedby$ ] Предположим, что граф удовлетворяет условию, но не имеет совершенного паросочетания. 

			Подставим пустое $S$ в условие: $o(G) \le \lvert \varnothing \rvert = 0$, то есть $v(G)$ четно.

			Пусть $G^*$ --- максимальный надграф $G$ на том же множестве вершин, не имеющий совершенного паросочетания. Хотим построить совершенное паросочетание в $G^*$, тем самым получив противоречие.

			Для любого $S \subset V(G)$ выполняется неравенство
			\[
			o(G^*-S) \le o(G-S) \le \lvert S \rvert
			.\] 

			Пусть $U = \{u \in V(G) \colon d_{G^*}(u) = v(G) -1\}$. Очевидно, что $G^*$ не может быть полным, поэтому $U \neq V(G)$.
			\begin{lemma}\label{lm:matching_2}
				Граф $G^* - U$ представляет собой объединение нескольких несвязных друг с другом полных подграфов.
			\end{lemma}
			\begin{proof}
			    \begin{itemize}
					\item Предположим противное. Тогда существуют такие вершины $x, y, z \in V(G) \setminus U$, что $xy, yz \in E(G^*)$, но $xz \notin E(G^*)$.
					\item Так как $y \notin U$, существует такая вершина $w \notin U$,  что $yw \notin E(G^*)$.
					\item Так как граф $G^*$ максимален, в графе $G^*+xz$ существует паросочетание $M_1$, а в графе $G^* + yw$ --- $M_2$. При этом $xz \in M_1$ и $yw \in M_2$, иначе в $G^*$ будет совершенное паросочетание.
					\item Пусть $H = (V(G), M_1 \vartriangle M_2)$. Граф  $H$ --- несвязное объединение четных циклов, в каждом из которых чередуются ребра из $M_1$ и $M_2$, поэтому в каждой компоненте есть совершенные паросочетания на ребрах $M_1$ и на ребрах $M_2$.
					\item Ребра $xz$ и $yw$ принадлежат ровно одному паросочетание, поэтому лежат и  $E(H)$.
			    \end{itemize}
				Разберем два случая:
				\begin{enumerate}
					\item Ребра $xz$ и $yw$ лежат в разных компонентах $C_1$ и $C_2$ графа $H$.

						Тогда можем выбрать на вершинах $C_1$ выбрать паросочетание из $M_2$, в $C_2$ из $M_1$, в остальных из любых. Так мы получили совершенное паросочетание в графе $G^*$. Противоречие. 
					\item Ребра $xz$ и $yw$ лежат в одной компоненте $C$ графа $H$.

						НУО, считаем, что в цикле $C$ вершины расположены в порядке $ywzx$.
\begin{figure}[ht]
    \centering
    \incfig{tutta-thm}
    \caption{Случай 2}
    \label{fig:tutta-thm}
\end{figure}
					Рассмотрим простой путь $P=xC''yzC'w$. Заметим, что $V(P) = V(C)$ и $E(P) \subset E(G^*)$. Следовательно, существует совершенное паросочетание $M_C \subset E(G^*)$ на вершинах компоненты связности $C$.

					В остальных компонентах можем выбрать ребра любого из $M_1$ и $M_2$. Так мы построили совершенное паросочетание графа $G^*$. Противоречие. 
				\end{enumerate}
			\end{proof}

			Будем использовать лемму. 

			\begin{itemize}
				\item Среди несвязных полных графов не более $\lvert U \rvert$ имеет нечетное число вершин по условию теоремы.
				\item В каждой четной компоненте графа $G^*-U$ мы построим полное паросочетание, в каждой нечетной --- паросочетание на всех вершинах кроме одной, оставшуюся мы соединим с вершиной из $U$. Мы используем различные вершины из $U$, их хватит.
				\item Наконец, разобьем на пары оставшиеся вершины из $U$ : это можно сделать, так как каждая из них смежна в $G^*$ со всеми остальными. 
			\end{itemize}
			Так мы построили совершенное паросочетание в графе $G^*$. Противоречие. 
    \end{description}
\end{proof}
