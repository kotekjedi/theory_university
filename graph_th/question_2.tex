\section{Существование гамильтонова пути и цикла: замыкание Хватала}

\begin{lemma}\label{lm:circle_2}
    Пусть вершины $a$ и $b$ не смежны и $d_G(a) + d_G(b) \ge v(G)$. Тогда граф $G$ гамильтонов, согда граф $G + ab$ тоже гамильтонов.
\end{lemma}
\begin{proof}
	\begin{itemize}
		\item Если $G$ гамильтонов, то и граф с дополнительным ребром $ab$ тоже гамильтонов.
		\item Докажем следствие в обратную сторону. Пусть граф $G + ab$ гамильтонов. 
			\begin{itemize}
				\item Если гамильтонов цикл не проходит по ребру $ab$, то он есть и в графе $G$. 
				\item Если проходит по $ab$, то в $G$ есть гамильтонов путь, причем сумма степеней его концов не меньше $v(G)$, тогда по лемме \ref{lm:circle_1} в графе $G$ есть гамильтонов цикл.
			\end{itemize}
	\end{itemize}    
\end{proof}

\begin{definition}[Замыкание графа]
	Рассмотрим произвольный граф $G$. Пока существуют две вершины $a, b \in V(G)$, для которых $d_G(a) + d_G(b) \ge v(G)$, добавим в граф соответствующее ребро $ab$. Полученный граф называется \selectedFont{замыканием} графа $G$, обозначается $C(G)$.
\end{definition}

\begin{corollary}[Хватал, 1974]
    Граф $G$ гамильтонов, согда его замыкание $C(G)$ --- гамильтонов граф.
\end{corollary}

\begin{lemma}[о единственности замыкания]\label{lm:circle_3}
    Замыкание графа $G$ определено однозначно, то есть не зависит от порядка добавления ребер.
\end{lemma}
\begin{proof}
    Пусть в результате двух различных цепочек добавления ребер были получены различные графы $G_1$ и $G_2$.

	Тогда есть ребра, добавленные при построении $G_1$, которых нет в $G_2$. Найдем такое ребро $ab$, которое было добавлено первым. 

	Обозначим граф, к которому мы добавили $ab$, за $G_0$. Тогда $d_{G_0}(a) + d_{G_0}(b) \ge v(G)$.

	С другой стороны, все ребра, добавленные к $G$ при построении $G_0$, добавлены и $G_2$. Поэтому, $d_{G_2}(a) + d_{G_2}(b) \ge v(G)$, следовательно, в $G_2$ нет ребра, которое мы должны были добавить. Противоречие. 
\end{proof}
