\section{Теорема Бородина о d-раскрасках}
\begin{definition}
    Граф, в котором каждый блок --- нечетный цикл или полный граф называется \selectedFont{лесом Галлаи}.
\end{definition}

\begin{theorem}[Бородин, 1977]
    Если $G$ не является лесом Галлаи, то $G$ $d$-раскрашиваем.
\end{theorem}
\begin{proof}
	Пусть каждой вершине $v$ соответствует список $L(v)$. НУО $l(v) = d_G(v)$ для каждой вершины $v \in V(G)$.  Считаем граф связными. 

	Индукция по размеру графа.
	\begin{description}
		\item[База:] $G$ двусвязен. 

			Если не все списки одинаковые, то существуют две смежные вершины $a$ и $b$ c $L(a) \neq L(b)$ и $G$ раскрашиваем по лемме \ref{lm:coloring_6}.

			Значит, все списки одинаковы, состоят из  $d$ цветов. Тогда и все степени вершин равны $d$.

			По условию граф отличается  от полного графа и нечетного цикла, поэтому, по теореме Брукса раскраска существует.
		\item [Переход:] $G$ недвусвязен.
			Пусть для меньшего чем $G$ графа теорема доказана. 

			Рассмотрим крайний блок $B$ графа $G$, отделяемый от остального графа точкой сочленения $a$. 

			Граф $B-a$ связен, все вершины нормальны по условию, а все смежные с $ a$ вершины избыточны, причем такие должны быть, иначе это не точка сочленения.
			По лемме \ref{lm:coloring_3} его вершины можно покрасить согласно спискам.

			Пусть $G' = G - \Int(B)$. Граф $G'$ имеет те же блоки, что и $G$, кроме $B$. Поэтому среди этих блоков должен быть еще один блок, отличный от нечетного цикла и полного графа.

			Списки отличных от $a$ вершин не менялись, степени --- тоже.

			Составим новый список $L'(a)$ из всех цветов $L(a)$,  кроме использованных для раскраски $\N_B(a)$. Таких цветов не более $d_B(a)$. Так как $d_G(a) = d_B(a) + d_{G'}(A)$, получаем $l'(a) \ge d_{G'}(a)$.

			По индукционному предположению существуют правильная раскраску вершин $G'$ в цвета из списка. Далее дополняем ее раскраской $B-a$ и получаем искомую раскраску вершин $G$.
	\end{description}
\end{proof}
