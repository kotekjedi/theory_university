\section{Максимальное паросочетание и дополняющие пути: теорема Бержа.}
\begin{definition}
    Пусть $M$ --- паросочетание в графе $G$.
	\begin{itemize}
		\item Назовем путь \selectedFont{М-чередующимся}, если в нем чередуются ребра из $M$ и ребра, не входящие в  $M$.
		\item Назовем $M$-чередующийся путь \selectedFont{M-дополняющим}, если его начало и конец не покрыты паросочетанием $M$.
	\end{itemize}
\end{definition}

\begin{theorem}[Берж, 1957]
    Паросочетание $M$ в графе $G$ является максимальным, согда нет $M$-дополняющих путей.
\end{theorem}
\begin{proof}
    \begin{description}
		\item[$ \implies$ ] Пусть в графе  $G$ существует $M$-дополняющий путь $S = a_1a_2\ldots a_{2k}$.

			Тогда мы можем заменить все входящие в $M$ ребра $a_2a_3, \ldots , a_{2k-2}a_{2k-1}$ на не входящие в $M$ ребра $a_1a_2, \ldots , a_{2k-1}a_{2k}$, увеличив паросочетание. Противоречие. 
		\item[$ \impliedby$] Пусть $M$ --- не максимальное паросочетание, тогда рассмотрим максимальное $M'$.

		Пусть $N = M \vartriangle M'$ и подграф $H = G(N)$. Для любой вершины $v \in  H$ имеем $d_H(v) \in \{1, 2\}$, поэтому $H$ --- объединение путей и циклов. 

		Причем в каждом пути или цикле ребра из $M$ и $M'$ чередуются. Так как ребер из $M'$ больше, есть хотя бы одна компонента $P$ графа $H$ --- путь нечетной длины, где ребер из $M'$ больше. Получается, что мы нашли $M$-дополняющий путь. Противоречие.
    \end{description}
\end{proof}
